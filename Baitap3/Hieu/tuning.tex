\documentclass{article}

\usepackage[utf8]{vietnam}
\usepackage{listings}
\usepackage{hyperref}

\title{Code Tuning và Optimization}

\begin{document}
    \section{Code Tuning}

    Trong bài tập 2, nhóm đã viết chương trình giải mô phỏng 
    thuật toán Merge Sort trên hai ngôn ngữ Java và Python,
    cũng như thực hiện các phương pháp kiểm thử tương ứng đối
    đối với hai mã nguồn này. Trong bài tập lần này, nhóm tiếp 
    tục sử dụng các mã nguồn trên để thực hiện tối ưu hóa hiệu
    năng của chương trình.

    \subsection{Một số thay đổi trước khi bắt đầu bài tập}

    Chương trình mô phỏng thuật toán Merge Sort bao gồm hai
    module là hàm merge() và hàm sort(). Cả hai hàm này đều được
    đặt quyền truy cập public, nhằm cho phép các chương trình 
    kiểm thử truy cập từ bên ngoài. 
    
    Tuy nhiên trong quá trình sử dụng, hàm merge() chỉ cần được 
    gọi như là một bước con của thuật toán Merge Sort. Do đó 
    trong bài tập lần này, quyền truy cập của merge() được đặt lại
    làm private. Điều này giúp giảm số lượng trường hợp cần phải
    xử lý, và làm cho ví dụ trở nên thực tế hơn.

    \subsection{Thực hiện các kỹ thuật hiệu chỉnh mã nguồn}
    
    \subsubsection{Hiệu chỉnh các biểu thức logic}
    Trong quá trình kiểm thử, hàm merge() và hàm sort() đều có các điều
    kiểm tra mảng được truyền vào. Trong trường hợp mảng truyền vào là null,
    hoặc có độ dài không quá 1, chương trình sẽ lập tức thoát ra, không phải
    xử lý gì cả.

    \begin{lstlisting}
    if (arr == null || arr.length <= 1)
        return arr;
    \end{lstlisting}

    Tuy nhiên đoạn code kiểm tra này chỉ nên được gọi một lần, khi mảng 
    được truyền vào lần đầu tiên. Trong các lần gọi tiếp theo của vòng lặp
    đệ quy, mảng được truyền vào được tính toán trực tiếp từ bên trong code,
    và không thể xảy ra trường hợp bằng null. 

    Ảnh hưởng về hiệu năng lên chương trình càng lớn hơn trong chương trình 
    Python: do Python không định kiểu chặt lên biến, các thủ tục phải thực
    hiện kiểm tra dữ liệu đầu vào có đúng kiểu hỗ trợ hay không, trước khi 
    tiếp tục xử lý. Việc kiểm tra này mất thời gian tuyến tính với độ dài 
    của mảng, và làm giảm hiệu suất chương trình một cách đáng kể:

    \begin{lstlisting}
    def __validate_(arr):
        if (type(arr) is not list):
            return False
        for x in arr:
            if (type(x) is not int):
                return False
        return True
    \end{lstlisting}

    \textbf{Phương án cải thiện:} Tạo thêm một thủ tục sortRecursive(), đảm nhiệm
    việc tính toán một lần đệ quy của thuật toán sắp xếp. Việc kiểm tra null
    và kiểm tra kiểu dữ liệu chỉ được thực hiện một lần duy nhất tại hàm sort().
    Sau khi kiểm tra xong, hàm sort() gọi sortRecursive(), thực hiện nội dung 
    chính của thuật toán. Điều này vừa giúp cải thiện hiệu năng mà không làm 
    thay đổi interface của chương trình.

    \begin{lstlisting}
    public static void sort(int[] arr) {
        if (arr == null)
            return;
        cache = new int[arr.length];
        sortRecursive(...);
    }
    \end{lstlisting}

    \subsubsection{Hiệu chỉnh vòng lặp}
    Trong hàm merge(), việc ghép hai nửa đã sắp xếp của mảng được thực hiện bằng
    cách duyệt qua tất cả các vị trí của mảng kết quả, và gán giá trị tại vị
    trí đó bằng giá trị nhỏ hơn trong hai giá trị đứng đầu mỗi nửa. Nếu như một
    trong hai nữa đã hết giá trị thì mặc định là gán bằng giá trị của nửa còn lại.

    \begin{lstlisting}
    for(int i = 0; i < result.length; i ++) {
        if (idLeft == left.length)
            result[i] = right[idRight ++];
        else if (idRight == right.length)
            result[i] = left[idLeft ++];
        else {
            if (left[idLeft] < right[idRight])
                result[i] = left[idLeft ++];
            else
                result[i] = right[idRight ++];
        }
    }
    \end{lstlisting}

    Trong trường hợp nửa bên phải hết trước, điều kiện kiểm tra vẫn phải được thực 
    hiện đối với nửa bên trái trước (và chắc chắn trả về đúng) khi thực hiện với 
    nửa bên phải. Điều này làm ảnh hưởng tới hiệu năng của chương trình.

    \textbf{Phương án cải thiện:} Thay một vòng lặp for bằng 3 vòng lặp while: một vòng
    lặp chạy khi cả hai nửa còn phần tử, và 2 vòng lặp chạy để đưa các phần tử còn 
    lại vào cuối kết quả:

    \begin{lstlisting}
    while(idLeft < middle && idRight < to) {
        if (cache[idLeft] < cache[idRight])
            arr[iterPos ++] = cache[idLeft ++];
        else
            arr[iterPos ++] = cache[idRight ++];
    }
    while(idLeft < middle)
        arr[iterPos ++] = cache[idLeft ++];
    while(idRight < to)
        arr[iterPos ++] = cache[idRight ++];
    \end{lstlisting}

    Điều này giúp loại bỏ công việc thừa thãi trong vòng lặp, làm tăng tốc độ chương 
    trình.

    \subsubsection{Hiệu chỉnh chuyển đổi dữ liệu}
    Do chương trình chỉ thao tác so sánh giữa các giá trị thuộc một kiểu dữ liệu duy 
    nhất (số nguyên) nên không phát sinh việc chuyển đổi dữ liệu.

    \subsubsection{Hiệu chỉnh các biểu thức}
    Thuật toán Merge Sort cần thực hiện tính toán vị trí chính giữa của các đoạn để 
    chia đoạn cần sắp xếp làm hai.

    \begin{lstlisting}
    int middle = arr.length / 2;
    \end{lstlisting}

    \textbf{Phương án cải thiện:} Phép toán chia một số nguyên cho hai được thực hiện rất nhiều lần, và có thế 
    được thay thế bằng phép dịch phải sang 1 để tăng cường hiệu năng.

    \begin{lstlisting}
    int middle = from + (length >> 1);
    \end{lstlisting}
    
    \subsubsection{Hiệu chỉnh hàm/thủ tục}
    Trong chương trình ban đầu, các thao tác tính toán trả về các object thuộc kiểu mảng số nguyên:
    việc chia mảng làm đôi được thực hiện bằng cách copy giá trị của mỗi nửa để tạo thành hai object
    mới; việc ghép hai mảng đã sắp xếp cũng trả về một object mảng, vân vân... Cách cài đặt này rất 
    tiện lợi cho việc đọc hiểu, vì chỉ sử dụng các hàm được cung cấp sẵn bởi ngôn ngữ, và phải thực 
    hiện rất ít thao tác toán. 
    
    \begin{lstlisting}
    int[] left = sort(
        Arrays.copyOfRange(arr, 0, middle)
    );
    int[] right = sort(
        Arrays.copyOfRange(arr, middle, arr.length)
    );
    int[] result = merge(left, right);
    \end{lstlisting}

    Tuy nhiên, việc tạo thêm object mới liên tục khiến cho mức tiêu thụ bộ nhớ của 
    thuật toán trở thành $O(nlog(n))$, cao hơn so với mức chuẩn là $O(n)$. Điều này vừa khiến chương 
    trình trở nên lãng phí tài nguyên hơn, lại vừa làm giảm hiệu năng do phải liên tục kêu gọi cấp phát 
    bộ nhớ cho các object mới.

    \textbf{Phương án cải thiện:} chỉnh sửa hàm sortRecursive() và merge() thành thao
    tác theo vị trí trên mảng được truyền vào, thay vì thao tác trên object. Ngoài ra trong hàm sort(),
    ta cũng khởi tạo một mảng số nguyên cache[] có cùng kích cỡ với mảng truyền vào để lưu trữ giá trị 
    trong khi chạy thuật toán. Nhờ đó, ta loại bỏ đi được việc cần phải yêu cầu bộ nhớ
    động.

    \begin{lstlisting}
    public static void sort(int[] arr) {
        if (arr == null)
            return;
        cache = new int[arr.length];
        sortRecursive(arr, 0, arr.length);
    }
    \end{lstlisting}


    \subsection{Kiểm tra cải thiện hiệu năng}
    \subsubsection{Môi trường kiểm thử}
    \begin{itemize}
        \item Hệ điều hành: Antergos Linux 
        \item Phiên bản Kernel: 5.0.5-arch1-1-ARCH
        \item Kiến trúc: 64-bit
        \item Vi xử lý: Intel® Core™ i3-3217U CPU, 4 nhân, 1.80GHz
        \item Bộ nhớ: 7.7 GiB of RAM
        \item Phiên bản Java: OpenJDK Runtime Environment 11.0.3
        \item Phieen bản Python: Python 3.7.3
    \end{itemize}

    Các chương trình đọc dữ liệu từ file test.txt, chứa một dãy các số
    nguyên được sinh ngẫu nhiên trong khoảng $[-10^9, 10^9]$. Chương 
    trình Java sẽ thực hiện sắp xếp dãy số $10^7$ phần tử, trong khi 
    chương trình Python sẽ thực hiện trên dãy số $10^6$ phần tử. Điều
    này là do Python là ngôn ngữ thông dịch, và do đó có tốc độ chậm hơn so
    với ngôn ngữ biên dịch là Java.

    Thời gian hoạt động của thuật toán được tính duy nhất trên hàm sort() -
    bỏ qua các giai đoạn nhập và xuất dữ liệu. Đối với Java, việc tính toán
    được thực hiện như sau:

    \begin{lstlisting}
    long startTime = Instant.now().toEpochMilli();
    sort(arr);
    long endTime = Instant.now().toEpochMilli();
    long runTime = endTime - startTime;
    System.out.println(runTime);    
    \end{lstlisting}

    Đối với Python:

    \begin{lstlisting}
    start_time = datetime.datetime.now()
    sort(array)
    end_time = datetime.datetime.now()
    print(end_time - start_time)
    \end{lstlisting}

    Nhóm cũng sử dụng \href{https://github.com/pshved/timeout}{timeout},
    một chương trình Perl nhỏ cho phép đo đạc lại thời gian hoạt động và 
    lượng tài nguyên sử dụng của chương trình command line trên Linux.

\end{document}