\documentclass{article}

\usepackage[utf8]{vietnam}

\title{Sử dụng debugger JDB và PDB cho Java và Python}
 
\begin{document}
    \section{Sử dụng debugger JDB và PDB cho Java và Python}

    Các ngôn ngữ lập trình thường được cung cấp các bộ công cụ hỗ trợ 
    debug. Các chức năng thường thấy trong các bộ công cụ này bao gồm:
    \begin{itemize}
        \item {Breakpoint (điểm dừng): Cho phép lập trình viên dừng 
        chương trình tại điểm xác định và theo dõi các giá trị tại điểm
        dừng hiện thời, từ đó suy luận ra lỗi. Điểm dừng có thể có hoặc
        không đi kèm điều kiện kích hoạt (conditional breakpoint).}
        \item {Stepping (thực hiện từng bước): Cho phép lập trình viên
        chạy chương trình theo từng bước một thay vì liên tục, cũng như
        theo dõi các giá trị ở từng bước.}
        \item {Stack Trace (truy vết ngăn xếp): Cho phép lập trình viên
        theo dõi call stack (ngăn xếp lệnh) để biết được lệnh nào đang
        gọi lệnh nào vào một thời điểm xác định.}
    \end{itemize}

    Hai ngôn ngữ Java và Python được cung cấp sẵn hai bộ công cụ debug
    mặc định là JDB và PDB.

    \subsection{JDB - Java Debugger}
    \textbf{jdb} là một bộ công cụ debug sử dụng command line dành cho Java.
    Nó hỗ trợ khả năng debug code Java chạy trên máy ảo nội bộ hoặc từ xa.

    \subsubsection{Khởi động một tiến trình jdb}
    Cú pháp chung để khởi động jdb trên giao diện command line là 
    \begin{verbatim}
        jdb [ options ] [ class ] [ arguments ]
    \end{verbatim}
    Trong đó:
    \begin{itemize}
        \item {options: các cài đặt cho tiến trình hiện tại.}
        \item {class: class để thực hiện debug.}
        \item {arguments: các tham số truyền vào chương trình, giống như khi
        khởi tạo máy ảo Java bình thường.}
    \end{itemize}
    
    Có hai cách thông dụng để tạo tiến trình jdb. Cách thứ nhất là yêu cầu jdb
    khởi tạo một máy ảo mới để chạy class cần được debug. Ví dụ:
    \begin{verbatim}
        jdb MyClass
    \end{verbatim}
    Khi được khởi tạo theo cách này, máy ảo Java sẽ tạm dừng trước khi bắt đầu
    thực hiện chỉ thị đầu tiên của chương trình.

    Cách thứ hai là kết nối jdb với một máy ảo đang chạy. Máy ảo đang chạy cần 
    phải sử dụng cài đặt:
    \begin{verbatim}
        -agentlib:jdwp=transport=dt_shmem,server=y,suspend=n
    \end{verbatim}
    để chuẩn bị các thư viện debug và chỉ thị cụ thể cách thức kết nối cho jdb. 
    Ví dụ: để khởi chạy class MyClass cho việc debug:
    \begin{verbatim}
        java -agentlib:jdwp=transport=dt_shmem,
            address=jdbconn,server=y,suspend=n MyClass
    \end{verbatim}
    Khi đó ta có thể kết nối với jdb bằng lệnh:
    \begin{verbatim}
        jdb -attach jdbconn 
    \end{verbatim}
    Chú ý rằng class MyClass không được truyền vào jdb, vì ta đang không khởi tạo máy
    ảo mới.

    \subsubsection{Các lệnh jdb cơ bản}
    Để sử dụng breakpoint, ta có một số cú pháp:
    \begin{itemize}
        \item {stop at MyClass:22 \textit{(Đặt breakpoint ở dòng thứ 22 của 
        file MyClass)}.}
        \item {stop in java.lang.String.length \textit{(Đặt breakpoint ở đầu của hàm\\
        java.lang.String.length)}.}
        \item {stop in MyClass.<init> \textit{(Đặt breakpoint tại đầu hàm khởi tạo của class MyClass)}.}
        \item {stop in MyClass.<clinit> \textit{(Đặt breakpoint tại đầu hàm khởi tạo static của class MyClass)}.}
    \end{itemize}
\end{document}